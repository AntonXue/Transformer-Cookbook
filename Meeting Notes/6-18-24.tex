\documentclass{article}
\usepackage{amssymb}
\usepackage{amsmath}
\usepackage{mathtools}
\usepackage{natbib}
\usepackage{url}
\usepackage[dvipsnames]{xcolor} %more colors for frame boxes

% colorful comments
\newcommand{\DA}[1]{({\textcolor{RedViolet}{\textbf{DA: #1}}})}
\newcommand{\LS}[1]{({\textcolor{MidnightBlue}{\textbf{LS: #1}}})}
\newcommand{\AY}[1]{({\textcolor{ForestGreen}{\textbf{AY: #1}}})}
\newcommand{\DC}[1]{({\textcolor{Dandelion}{\textbf{DC: #1}}})}
\newcommand{\WM}[1]{({\textcolor{RoyalPurple}{\textbf{WM: #1}}})}
\newcommand{\uvp}[1]{({\textcolor{Cyan}{\textbf{YP: #1}}})}
\newcommand{\GF}[1]{({\textcolor{Salmon}{\textbf{GF: #1}}})}

\newcommand{\AX}[1]{({\textcolor{blue}{\textbf{AX: #1}}})}
\newcommand{\BD}[1]{({\textcolor{BlueGreen}{\textbf{BD: #1}}})}
\newcommand{\EF}[1]{({\textcolor{gray}{\textbf{EF: #1}}})}

\newcommand{\response}[1]{{\textcolor{red}{\textbf{#1}}}}

\title{Cooking Session}
\date{June 18, 2024}
\begin{document}
\maketitle

\section{Attendees}

Emile, Andy, Will

\section{Today's Agenda}
\begin{itemize}
    \item Audience
    \item Positional Encodings
\end{itemize}

\section{Notes}
\begin{itemize}
    \item What is our audience like?
        \begin{itemize}
            \item Undergraduates or graduate students interested in transformers
            \item An ML engineer interested in the theory
            \item This could even be the basis of an undergraduate course. Or, like a module inside of one.
            \item We would like this to be useful for people outside of FLaNN. For instance, an NLP practitioner looking to gain some more insight into what transformers can do.
            \item It is useful when introducing students to transformers to have something tangible they can work through. Like RASP. But the idea is to work through the constructions, to get intuition on what transformers can do.
            \item This is after all, a writeup all about lower bounds.
            \item It's one thing to call the pytorch functions to implement transformers, it's another to work through a construction by hand. For instance, a construction to just attend to the previous token. It's more satisfying, and you gain a deeper intuition by going through it manually.
            \item We could have exercises, and flesh this out into more of a textbook-like thing.
            \item This could be an updated and more theoretically grounded version of Sasha's RASP exercises. But using S-RASP, perhaps. People like RASP.
        \end{itemize}
    \item Have applications of the attention constructions. Like, table lookup can be used for XYZ.
    \item Exercises: For instance, Dyck-2. These can be interspersed or at the end of the chapter.
\end{itemize}

TO DO;
\begin{itemize}
    \item Find positional encodings, make a PR to add to the section
    \item Change to "attention", talk about other kinds of attention besides self-attention Andy can do
\end{itemize}

\bibliography{main}
\bibliographystyle{plain}
\end{document}


