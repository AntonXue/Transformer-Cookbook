\usepackage{amssymb}
\usepackage{amsmath}
\usepackage{mathtools}
\usepackage{mleftright}
\usepackage{natbib}
\usepackage[gen]{eurosym}
\usepackage{url}

% tables and such
\usepackage{blkarray}
\usepackage{colortbl}

\usepackage[dvipsnames]{xcolor} %more colors for frame boxes

\usepackage[english]{babel}
% \usepackage{textcomp}

\addtokomafont{paragraph}{\rmfamily}

\usepackage[breakable,skins]{tcolorbox}
\tcbuselibrary{theorems}

\usepackage[hidelinks]{hyperref}
\usepackage[capitalize,nameinlink]{cleveref} % should go after \tcbuselibrary and \usepackage{hyperref} but before \newtcbtheorem

% tikz stuff
\usepackage{tikz}
\usetikzlibrary{decorations.markings}
\usetikzlibrary{arrows,positioning,shapes,decorations,automata,petri,backgrounds,arrows.meta}
\usetikzlibrary{calc}
\usetikzlibrary{fit}
\usepackage{pgfplots}
\usepackage{dirtytalk}  % use \say{} for quotes
\usepackage{booktabs}

% David's NN package
\usepackage{nn}

% % Andy slightly prefers dashes instead of dots, if nobody else cares
% \renewcommand{\labelitemi}{--}

\clubpenalty=10000
\widowpenalty=10000
\displaywidowpenalty=10000

\newcommand{\blank}{\text{\textvisiblespace}}
\newcommand{\N}{\mathbb{N}}
\newcommand*{\bqed}{\hfill\ensuremath{\blacksquare}}

\tcbset{theorembox/.style={%
    skin=standard jigsaw,% Better drawing engine
    breakable,% Allow breaking box on two pages
    parbox=false,% Make text flow like normal text
    boxrule=0mm,% In general no border
    leftrule=1mm,% Only left border
    boxsep=2mm,% General padding around text
    arc=0mm,% No rounded border
    outer arc=0mm,% No rounded border
    left=0mm,% Additional padding left of text
    right=0mm,% Additional padding right of text
    top=0mm,% Additional padding above text
    bottom=0mm,% Additional padding below text
    oversize,% Box content should be as wide as normal text
    enlarge top initially by=1mm,% Add some margin before the box starts
    description delimiters parenthesis,
    separator sign none,
    fonttitle=\bfseries,
    description font=\mdseries,
    terminator sign=.,
    coltitle=black,
}}

\colorlet{Light-CadetBlue}{CadetBlue!50!white}

\newtcbtheorem[crefname={definition}{Definition},Crefname={Definition}{Definition}]{definition}{Definition}{theorembox,colframe=RoyalBlue,colback=RoyalBlue!3,colbacktitle=RoyalBlue!3}{definition}

\newtcbtheorem[]{prf}{Proof}{theorembox,colframe=CadetBlue,colback=CadetBlue!3,colbacktitle=CadetBlue!3}{prf}

\newtcbtheorem[crefname={claim}{Claim},Crefname={Claim}{Claim}]{clm}{Claim}{theorembox,colframe=CarnationPink,colback=CarnationPink!3,colbacktitle=CarnationPink!3}{clm}

\newtcbtheorem[crefname={theorem}{Theorem},Crefname={Theorem}{Theorem}]{theorem}{Theorem}{theorembox,colframe=DarkOrchid,colback=DarkOrchid!3,colbacktitle=DarkOrchid!3}{thm}

\newtcbtheorem[crefname={corollary}{Corollary},Crefname={Corollary}{Corollary}]{cor}{Corollary}{theorembox,colframe=Dandelion,colback=Dandelion!3,colbacktitle=Dandelion!3}{cor}

\newtcbtheorem[crefname={example}{Example},Crefname={Example}{Example}]{exmp}{Example}{theorembox,colframe=ForestGreen,colback=ForestGreen!3,colbacktitle=ForestGreen!3}{exmp}

\newtcbtheorem[]{rem}{Remark}{theorembox,colframe=Peach,colback=Peach!3,colbacktitle=Peach!3}{rem}

% Transformer component shortcuts
\newcommand{\ReLU}{\text{ReLU}}
\newcommand{\GELU}{\text{GELU}}

% RASP shortcuts
% B-RASP
\DeclareMathOperator{\leftmost}{{\mathbf{\blacktriangleleft}}}
\DeclareMathOperator{\rightmost}{{\mathbf{\blacktriangleright}}}
\newcommand{\BRASP}{\textbf{B-RASP}}
\newcommand{\att}[5]{\ensuremath{{#1}_{#2} \left[#3, #4\right] \; #5}}
\newcommand{\attl}[4]{\att{\leftmost}{#1}{#2}{#3}{#4}}
\newcommand{\attr}[4]{\att{\rightmost}{#1}{#2}{#3}{#4}}
\newcommand{\longatt}[5]{\begin{aligned}[t]{#1}_{#2} &\left[#3, #4\right] \\ &#5\end{aligned}}
\newcommand{\longattl}[4]{\longatt{\leftmost}{#1}{#2}{#3}{#4}}
\newcommand{\longattr}[4]{\longatt{\rightmost}{#1}{#2}{#3}{#4}}
% C-RASP
\newcommand{\cttn}[2]{\ensuremath{\textsc{\textbf{\#}} \left[ #1 \right] \; #2}}
\newcommand{\cif}[3]{\ensuremath{#1\;\mathbf{?}\; #2\; \textbf{:} \;#3}}
\newcommand{\cifop}{\ensuremath{\mathbf{?}}}


% Math shortcuts
\def\R{\mathbb{R}}
\def\N{\mathbb{N}}
\def\Z{\mathbb{Z}}
\def\C{\mathbb{C}}
\def\Q{\mathbb{Q}}
\def\Ccal{\mathcal{C}}
\def\Rcal{\mathcal{R}}
\def\Fcal{\mathcal{F}}
\def\Lcal{\mathcal{L}}
\def\Mcal{\mathcal{M}}
\def\Ncal{\mathcal{N}}
\def\F{\mathbb{F}}

% Groupings
\DeclarePairedDelimiter\parens{(}{)}
\DeclarePairedDelimiter\bracks{[}{]}
\DeclarePairedDelimiter\braces{\{}{\}}
\DeclarePairedDelimiter\angles{\langle}{\rangle}
\DeclarePairedDelimiter\abs{\lvert}{\rvert}
\DeclarePairedDelimiter\norm{\lVert}{\rVert}
\DeclarePairedDelimiter\ceil{\lceil}{\rceil}
\DeclarePairedDelimiter\floor{\lfloor}{\rfloor}

% Fonts in math mode
\newcommand\mbb[1]{\mathbb{#1}}
\newcommand\mbf[1]{\mathbf{#1}}
\newcommand\mcal[1]{\mathcal{#1}}
\newcommand\mrm[1]{\mathrm{#1}}
\newcommand\msf[1]{\mathsf{#1}}

% colorful comments 
\newcommand{\DA}[1]{({\textcolor{RedViolet}{\textbf{DA: #1}}})}
\newcommand{\LS}[1]{({\textcolor{MidnightBlue}{\textbf{LS: #1}}})}
\newcommand{\AY}[1]{({\textcolor{ForestGreen}{\textbf{AY: #1}}})}
\newcommand{\DC}[1]{({\textcolor{Dandelion}{\textbf{DC: #1}}})}
\newcommand{\WM}[1]{({\textcolor{RoyalPurple}{\textbf{WM: #1}}})}
\newcommand{\uvp}[1]{({\textcolor{Cyan}{\textbf{YP: #1}}})}
\newcommand{\GF}[1]{({\textcolor{Salmon}{\textbf{GF: #1}}})}

\newcommand{\AX}[1]{({\textcolor{blue}{\textbf{AX: #1}}})}
\newcommand{\BD}[1]{({\textcolor{BlueGreen}{\textbf{BD: #1}}})}
\newcommand{\EF}[1]{({\textcolor{gray}{\textbf{EF: #1}}})}

\newcommand{\Instruction}[1]{{\textcolor{red}{\textbf{Instruction:} #1}}}
