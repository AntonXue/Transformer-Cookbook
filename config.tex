\usepackage{amssymb}
\usepackage{amsmath}
\usepackage{mathtools}
\usepackage{natbib}

\usepackage[gen]{eurosym}

\usepackage[dvipsnames]{xcolor} %more colors for frame boxes

\usepackage[english]{babel}

\addtokomafont{paragraph}{\rmfamily}

\usepackage[breakable,skins]{tcolorbox}
\tcbuselibrary{theorems}

\usepackage[hidelinks]{hyperref}
\usepackage[capitalize,nameinlink]{cleveref} % should go after \tcbuselibrary and \usepackage{hyperref} but before \newtcbtheorem

% tikz stuff
\usepackage{tikz}
\usetikzlibrary{decorations.markings}
\usetikzlibrary{arrows,positioning,shapes,decorations,automata,petri,backgrounds,arrows.meta}
\usetikzlibrary{calc}
\usetikzlibrary{fit}
\usepackage{pgfplots}
\usepackage{dirtytalk}

% David's NN package
\usepackage{nn}

\clubpenalty=10000
\widowpenalty=10000
\displaywidowpenalty=10000

\newcommand{\blank}{\text{\textvisiblespace}}
\newcommand{\N}{\mathbb{N}}
\newcommand*{\bqed}{\hfill\ensuremath{\blacksquare}}

\tcbset{theorembox/.style={%
    skin=standard jigsaw,% Better drawing engine
    breakable,% Allow breaking box on two pages
    parbox=false,% Make text flow like normal text
    boxrule=0mm,% In general no border
    leftrule=1mm,% Only left border
    boxsep=2mm,% General padding around text
    arc=0mm,% No rounded border
    outer arc=0mm,% No rounded border
    left=0mm,% Additional padding left of text
    right=0mm,% Additional padding right of text
    top=0mm,% Additional padding above text
    bottom=0mm,% Additional padding below text
    oversize,% Box content should be as wide as normal text
    enlarge top initially by=1mm,% Add some margin before the box starts
    description delimiters parenthesis,
    separator sign none,
    fonttitle=\bfseries,
    description font=\mdseries,
    terminator sign=.,
    coltitle=black,
}}

\colorlet{Light-CadetBlue}{CadetBlue!50!white}

\newtcbtheorem[crefname={definition}{Definition},Crefname={Definition}{Definition}]{definition}{Definition}{theorembox,colframe=RoyalBlue,colback=RoyalBlue!3,colbacktitle=RoyalBlue!3}{definition}

\newtcbtheorem[]{prf}{Proof}{theorembox,colframe=CadetBlue,colback=CadetBlue!3,colbacktitle=CadetBlue!3}{prf}

\newtcbtheorem[crefname={claim}{Claim},Crefname={Claim}{Claim}]{clm}{Claim}{theorembox,colframe=CarnationPink,colback=CarnationPink!3,colbacktitle=CarnationPink!3}{clm}

\newtcbtheorem[crefname={theorem}{Theorem},Crefname={Theorem}{Theorem}]{theorem}{Theorem}{theorembox,colframe=DarkOrchid,colback=DarkOrchid!3,colbacktitle=DarkOrchid!3}{thm}

\newtcbtheorem[crefname={corollary}{Corollary},Crefname={Corollary}{Corollary}]{cor}{Corollary}{theorembox,colframe=Dandelion,colback=Dandelion!3,colbacktitle=Dandelion!3}{cor}

\newtcbtheorem[crefname={example}{Example},Crefname={Example}{Example}]{exmp}{Example}{theorembox,colframe=ForestGreen,colback=ForestGreen!3,colbacktitle=ForestGreen!3}{exmp}

\newtcbtheorem[]{rem}{Remark}{theorembox,colframe=Peach,colback=Peach!3,colbacktitle=Peach!3}{rem}

% Transformer component shortcuts
\newcommand{\ReLU}{\text{ReLU}}

% Math shortcuts
\def\R{\mathbb{R}}
\def\N{\mathbb{N}}
\def\Z{\mathbb{Z}}
\def\C{\mathbb{C}}
\def\Q{\mathbb{Q}}
\def\Ccal{\mathcal{C}}
\def\Rcal{\mathcal{R}}
\def\Fcal{\mathcal{F}}
\def\Lcal{\mathcal{L}}
\def\Mcal{\mathcal{M}}
\def\Ncal{\mathcal{N}}
\def\F{\mathbb{F}}

% colorful comments
\newcommand{\DA}[1]{({\textcolor{RedViolet}{\textbf{DA: #1}}})}
\newcommand{\LS}[1]{({\textcolor{MidnightBlue}{\textbf{LS: #1}}})}
\newcommand{\AY}[1]{({\textcolor{ForestGreen}{\textbf{AY: #1}}})}
\newcommand{\DC}[1]{({\textcolor{Dandelion}{\textbf{DC: #1}}})}
\newcommand{\WM}[1]{({\textcolor{RoyalPurple}{\textbf{WM: #1}}})}
\newcommand{\uvp}[1]{({\textcolor{Cyan}{\textbf{YP: #1}}})}
\newcommand{\GF}[1]{({\textcolor{Salmon}{\textbf{GF: #1}}})}

\newcommand{\Instruction}[1]{{\textcolor{red}{\textbf{Instruction:} #1}}}
