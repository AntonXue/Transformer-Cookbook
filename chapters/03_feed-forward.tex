%-----------------------------------
%
\chapter{Feed-Forward Layers}
%
%-----------------------------------


Feed-Forward layers are an essential component of any good transformer!
Empirically, they have been observed to contribute to the expressive power transformer models \cite{geva-etal-2021-transformer}.
Here, we give a theoretical treatment of the feed-forward layer, and show how it can be used to compute a variety of important functions.
Although given enough parameters and a large enough input and hidden dimension, a feed-forward layer can approximate many functions \cite{hornik1989multilayer}, we focus on those that can be expressed exactly.


Much of the expressive power of feed-forward layers comes from their non-linear \emph{activation functions}, of which there are many. We define two popular activation functions. Other notable activation functions include sigmoid, softmax, hyperbolic tangent, Exponential Linear Unit (ELU) and Gated Linear Unit (GLU).

\begin{definition}{Pointwise function application}{}
  For any function $f \colon \R^d \to \R^d$ and any vector $\mathbf{x} = [x_i]_{i=1}^d \in \R^d$, we write
  \begin{equation*}
    f\mleft(\begin{bmatrix} x_1 \\ x_2 \\ \vdots \\ x_d \end{bmatrix}\mright) =  \begin{bmatrix} f(x_1) \\ f(x_2) \\ \vdots \\ f(x_d) \end{bmatrix}.
  \end{equation*}
\end{definition}

\begin{definition}{ReLU}{}
    $\ReLU : \R^d\to \R^d$, or Rectified Linear Unit, is a non-linear activation function as defined below \cite{Fukushima1975}.

    \[\ReLU(x) = \max(0,x).\]
\end{definition}

\begin{definition}{GELU}{}
    Gaussian Error Linear Unit (GELU) is a non-linear activation function. $\GELU : \R^d\to \R^d$ is computed as follows, where $\Phi$ is the cumulative distribution function of the standard normal distribution and $\sigma(x) = 1/(1+e^{-x})$ \cite{hendrycks2023}.

    \[\GELU(x) = x\,\Phi(x) \approx \sigma(1.702 x).\]
\end{definition}
\DC{This definition doesn't agree with the one in \cref{sec:ffnn_multiplication}.}

\begin{definition}{Feed-Forward Layer}{hidden text?}
    A feed-forward network is a function $f:\R^d\to \R^d$ that computes
    \[f(\mathbf{x}) = L_2(\ReLU(L_1(\mathbf{x})))\]
    where $L_1: \R^d \to \R^{d_{\text{ff}}}$ and $L_2: \R^{d_{\text{ff}}} \to \R^{d}$ are affine transformations. $L_1$ uses linear transformation $W_1:\R^d\to \R^{d_{\text{ff}}}$ and bias $b_1\in\R^{d_{\text{ff}}}$
    \[L_1(\mathbf{x}) = W_1 \mathbf{x} + b_1\]
    and $L_2$ uses linear transformation $W_2:\R^{d_{\text{ff}}}\to \R^d$ and bias $b_2\in\R^{d}$
    \[L_2(\mathbf{x}) = W_2 \mathbf{x} + b_2.\]
\end{definition}

\section{Continuous Piecewise Linear Functions}

\begin{definition}{continuous piecewise linear}{}
  Let $X \subseteq \R^d$. A function $f \colon X \to \R$ is called \emph{continuous piecewise linear (CPWL)} if there are closed subsets $X_1, \ldots, X_n \subseteq X$ such that $\bigcup_{i\in [n]}  X_i = X$, and for all $i \in [n]$, $\left.f\right|_{X_i}$ is affine.
\end{definition}

For the univariate case ($d=1$), suppose we want to construct a FFNN that computes a CPWL function \(f \colon \mathbb{R} \to \mathbb{R}\). Assume that $f$ has $n \ge 2$ pieces (of which the first and last extend to infinity) and is represented by points \(x_1 < \ldots < x_{n+1}\) and values \(y_1, \ldots, y_{n+1}\) such that \(f(x_k) = y_k\) for \(k = 1, \ldots, n+1\).
To do this, we first define the slopes \(m_1, \ldots, m_n\):
\begin{align*}
  m_k &= \frac{y_{k+1} - y_{k}}{x_{k+1} - x_{k}} \enskip \text{for} \enskip k = 1, \ldots, n.
\end{align*}
Then, \(f\) is computed by the following two-layer neural network:
\begin{align*}
    f(x) &= y_1 + m_1 (x-x_1) + \sum_{k=2}^{n} (m_k - m_{k-1}) \, \ReLU(x - x_k) \\
    &= \begin{dcases}
        y_1 + m_1 (x - x_1) &x \leq x_2 \\
        y_k + m_{k} (x - x_k) & x_k \leq x \leq x_{k+1} \\
        y_n + m_n (x - x_n) & x \geq x_n.
    \end{dcases}
\end{align*}

For the multivariate case ($f \colon \R^d \to \R$), any CPWL function with $k$ pieces can be computed exactly by a FFNN with $O(\log k)$ layers \citep{arora+:2018}.

\section{Identity Function}\label{sec:ffnn_identity}

This one is very easy: zero out the FFNN itself and let the residual connection do all the work.
\begin{align*}
W_1 = W_2 &= \mathbf{0} \\
b_1 = b_2 &= 0.
\end{align*}

\section{Canceling Residual Connections}\label{sec:ffnn_cancel_residual}

    \begin{tabular}{|p{1.5cm}|p{1.5cm}|}
        \hline
        \rowcolor{orange!20} % Set header color here
        \textbf{Input} & \textbf{Output} \\
        \hline
        $\mathbb{R}^d$ & $\mathbb{R}^d$ \\
        \hline
    \end{tabular}

    We can cancel out residual connections in FFNs. If $f\colon\R^d\to \R^d$ is a FFN with parameters $W_1, b_1, W_2, b_2$, there is a FFN $f'$ with parameters
    \begin{align*}
        W_1'&=\begin{bmatrix}
            W_1\\
            \textbf{I}\\
            -\textbf{I}
        \end{bmatrix}  &\quad b_1'&=\begin{bmatrix}
            b_1\\
            \textbf{0}\\
            \textbf{0}
        \end{bmatrix}\\
        W_2'&=\begin{bmatrix}
            W_2& -\textbf{I} & \textbf{I}
        \end{bmatrix}
        & \quad b_2' &= b_2
    \end{align*}
    so that $f'(\mathbf{x}) + \mathbf{x} = f(\mathbf{x})$, that is, it behaves like $f$ without a residual connection \citep{chiang+:icml2023}.

\section{Min and Max}\label{sec:ffnn_minmax}
    \begin{tabular}{|p{1.5cm}|p{1.5cm}|}
        \hline
        \rowcolor{orange!20} % Set header color here
        \textbf{Input} & \textbf{Output} \\
        \hline
        $\R\times\R$ & $\R$ \\
        \hline
    \end{tabular}


    Recall that $\ReLU(x)=\max(0,x)$. Then  $\min(x,y) = x-\ReLU(x-y)$.
    % $= y-\ReLU(y-x)$. To see this, there are two cases:
    \begin{itemize}
        \item If $x<y$, then $\ReLU(x-y)=0$, so $x-\ReLU(x-y)=x = \min(x,y)$.
        \item If $x\geq y$, then $\ReLU(x-y)=x-y$, so $x-\ReLU(x-y)=x-(x-y)=y = \min(x,y)$.
    \end{itemize}
    Similarly, $\max(x,y) = x+\ReLU(y-x)$. Therefore there exist FFNs to compute the min or max of two real numbers:
    \begin{center}
    \begin{tabular}{c@{\hspace*{4em}}c}
      \begin{tikzpicture}[x=1.5cm,y=1.5cm,baseline=1cm]
        \node (x1) at (0,0) [input,label=below:{$x$}];
        \node (x2) at (1,0) [input,label=below:{$y$}];
        \node (h1) at (-0.5,1) [relu] edge node[near start] {$1$} (x1);
        \node (h2) at (0.5,1) [relu] edge node {$-1$} (x1);
        \node (h3) at (1.5,1) [relu] edge node[near start] {$1$} (x1) edge node[auto=left,near start] {$-1$} (x2);
        \node (y) at (0.5,2) [output,label=above:{$\min(x,y)$}] edge node {$1$} (h1) edge node[auto=left] {$-1$} (h2) edge node[auto=left] {$-1$} (h3);
      \end{tikzpicture} &
      \begin{tikzpicture}[x=1.5cm,y=1.5cm,baseline=1cm]
        \node (x1) at (0,0) [input,label=below:{$x$}];
        \node (x2) at (1,0) [input,label=below:{$y$}];
        \node (h1) at (-0.5,1) [relu] edge node[near start] {$1$} (x1);
        \node (h2) at (0.5,1) [relu] edge node {$-1$} (x1);
        \node (h3) at (1.5,1) [relu] edge node[near start] {$-1$} (x1) edge node[auto=left,near start] {$1$} (x2);
        \node (y) at (0.5,2) [output,label=above:{$\max(x,y)$}] edge node {$1$} (h1) edge node[auto=left] {$-1$} (h2) edge node[auto=left] {$1$} (h3);
      \end{tikzpicture}
    \end{tabular}
    \end{center}


\section{Addition and Subtraction}\label{sec:ffnn_addition}

    \begin{tabular}{|p{1.5cm}|p{1.5cm}|}
        \hline
        \rowcolor{orange!20} % Set header color here
        \textbf{Input} & \textbf{Output} \\
        \hline
        $\R^d$ & $\R$ \\
        \hline
    \end{tabular}

    If you have a tensor of dimensions $\R^{d\times n}$, and you want to add
    $d_1$ and $d_2$ into $d_3$, then use FFN with $W_1:\R^d\to \R^4$ and $W_2:\R^4\to \R^d$

    \begin{equation*}
        W_1=
        \begin{blockarray}{cccccccc}
            & & d_1 & & d_2 & & d_3 & \\
            \begin{block}{c[ccccccc]}
                    1& \cdots & 1 & \cdots & 0 & \cdots & 0 & \cdots \\
                    2& \cdots & -1 & \cdots & 0 & \cdots & 0 & \cdots \\
                    3& \cdots & 0 & \cdots & 1 & \cdots & 0 & \cdots \\
                    4& \cdots & 0 & \cdots & -1 & \cdots & 0 & \cdots \\
            \end{block}
        \end{blockarray}
    \end{equation*}

    Regardless whether the values in $d_1$ and $d_2$ are positive or negative, after applying $\ReLU$ the resulting tensor $\begin{bmatrix}
        \ReLU(A_{d_1,*})\phantom- \\
        \ReLU(-A_{d_1,*})\\
        \ReLU(A_{d_2,*})\phantom- \\
        \ReLU(-A_{d_2,*})\\
    \end{bmatrix}$ will store a copy of $|x|$ in the even dimensions if $x$ was positive, and the even dimension if $x$ was negative. Using this, we can construct $W_2$ to add the correct original values together into $d_3$

    \begin{equation*}
    W_1=
    \begin{blockarray}{ccccc}
        & 1& 2& 3&4 \\
        \begin{block}{c[cccc]}
                & \vdots &\vdots & \vdots & \vdots \\
                d_1 & 0 &\ 0 & 0 & 0 \\
                & \vdots &\vdots & \vdots & \vdots \\
                d_2 & 0 & 0 & 0 & 0 \\
                & \vdots &\vdots & \vdots & \vdots \\
                d_3 & 1  & -1 & 1 & -1 \\
                & \vdots &\vdots & \vdots & \vdots \\
        \end{block}
    \end{blockarray}
    \end{equation*}

    We assume $d_3$ originally held $0$'s, or we can cancel the residual connection as described in \cref{sec:ffnn_cancel_residual}.

\section{Boolean Functions}\label{sec:ffnn_boolean}

    \begin{tabular}{|c|c|}
        \hline
        \rowcolor{orange!20} % Set header color here
        \textbf{Input} & \textbf{Output} \\
        \hline
        $\left\{\begin{bmatrix}
            -\delta\\\phantom- \delta1
        \end{bmatrix}, \begin{bmatrix}
            \phantom- \delta\\-\delta
        \end{bmatrix}\right\}$ & $\left\{\begin{bmatrix}
            -\delta\\\phantom- \delta
        \end{bmatrix}, \begin{bmatrix}
            \phantom- \delta\\-\delta
        \end{bmatrix}\right\}$ \\
        \hline
    \end{tabular}

    In this section, we show how to compute arbitrary Boolean functions using a single feed-forward network.
    This presentation was found in \cite{yang2024counting} \AY{Maybe there's an older one?}
    We show the construction for a zero-mean Boolean representation (as discussed in \cref{sec:booleans}).
    It is straightforwardly adapted to other representations.

    First, a single $\land$ and $\lnot$ can be computed by FFNNs with $\ReLU$ activations. Conjunction ($\land$) is equivalent to min/max:

    \begin{align*}
    \begin{bmatrix} \vdots \\ -2B_1+\delta \\ \phantom{-}2B_1-\delta \\ \vdots \end{bmatrix} \land \begin{bmatrix} \vdots \\ -2B_2+\delta \\ \phantom{-}2B_2-\delta \\ \vdots \end{bmatrix} &= \begin{bmatrix} \vdots \\ \max(-2B_1+\delta,-2B_2+\delta) \\ \min(\phantom{-}2B_1-\delta,\phantom{-}2B_2-\delta) \\ \vdots \end{bmatrix}.
    \end{align*}

    Logical negation ($\lnot$) is equivalent to arithmetic negation, or swapping the positive and negative components:
    \begin{center}
        \begin{tikzpicture}[x=2cm,y=1.5cm,baseline=1cm]
        \node (x1) at (0,0) [input,label=below:{$2B-\delta$}];
        \node (x2) at (1,0) [input,label=below:{$-2B+\delta$}];
        \node (h1) at (0,1) [relu] edge node {$1$} (x1);
        \node (h2) at (1,1) [relu] edge node {$1$} (x2);
        \node (y1) at (0,2) [output] edge node {$1$} (h1) edge node[near start] {$-1$} (h2);
        \node (y2) at (1,2) [output] edge node[auto=left,near start] {$-1$} (h1) edge node[auto=left] {$1$} (h2);
        \end{tikzpicture}
    \end{center}

    For an arbitrary Boolean formula, it can be more involved. First convert it to \emph{canonical} disjunctive normal form, which is a disjunction $\phi_1 \lor \cdots \lor \phi_n$ of clauses, at most one of which can be true for any value of the inputs.
    Each clause is of the form $\phi_m = \bigwedge_{k\in K_m} \psi_k$, where each $\phi_k$ is an input or a negated input and $K_m$ is a set of indices for each clause. A slightly different construction can be used to compute $\land$ over inputs in $K_m$. Observe that:

    \begin{align*}
    \bigwedge_{k\in K_m} B_k &= \ReLU\left(\left(\sum_{k\in K_m} (B_k)\right) -(|K_m|-1)\right) \\ &=\ReLU\left(\left(\sum_{k\in K_m} \frac{1}{2}(2B_k-\delta) \right) - \frac{3|K_m|}{2}+1\right).
    \end{align*}

    And this can be computed for each clause using the first layer and $\ReLU$ of a feed-forward layer.
    Note that the value of $\delta$ can be stored in every feature vector using an appropriate word embedding, as described \AY{Somewhere}.
    Then, because at most one clause can be true, the sum of all clauses will either be $1$ or $0$. Then, we convert back to the $\pm\delta$ representation of truth values.

    \[\bigvee_{m=1}^n \left(\bigwedge_{k\in K_m} B_k \right)=2\cdot\left(\sum_{m=1}^n\ReLU\left(\left(\sum_{k\in K_m} \frac{1}{2}(2B_k-\delta) \right) - \frac{3|K_m|}{2}+1\right)\right)-\delta.\]

    This can all be done in a single feed-forward layer.


\section{Conditionals}\label{sec:ffnn_conditional}

    \begin{tabular}{|c|p{1.5cm}|}
        \hline
        \rowcolor{orange!20} % Set header color here
        \textbf{Input} & \textbf{Output} \\
        \hline
        $\{0,1\}\times \R\times\R$ & $\R$ \\
        \hline
    \end{tabular}

    We assume the input representation to a feedforward net (after layer-norm) contains three values: a mask $p \in \{0, 1\}$, a first option $\phi$, and second option $\psi$. The goal is to compute the ``ternary'' expression:
    \begin{equation*}
        \cif{p}{\phi}{\psi} =
        \begin{cases}
            \phi & \textrm{if} \; p \\
            \psi & \textrm{otherwise.}
        \end{cases}
    \end{equation*}

    We adapt a construction used in \citep{merrill2024the} (Theorem 1, arXiv version) for two-layer ReLU feedforward nets.
    Let $\vec 1$ be a vector of ones.
    In the first layer, we define two neurons:
    \begin{align*}
        h_1 &= \ReLU(-p \vec 1 + \vec 1 + \phi) \\
        h_2 &= \ReLU(p \vec 1 - \vec 1 + \psi) .
    \end{align*}
    The second layer then simply computes $h_1 + h_2$, which by construction computes $\cif{p}{\phi}{\psi}$.

    \WM{Can justify this by writing out cases when $p = 0$ or $p = 1$}


\section{Multiplication}
\label{sec:ffnn_multiplication}

By definition~\footnote{\url{https://pytorch.org/docs/stable/generated/torch.nn.GELU.html}}:
\begin{align*}
    \mrm{GeLU}(x) \equiv \frac{x}{2} \parens*{1 + \tanh\parens*{\sqrt{\frac{2}{\pi}} (x + 0.044715x^3)}}
\end{align*}
and so from~\citep[Lemma 4]{akyurek2022learning}:
\begin{align*}
    \sqrt{\frac{\pi}{2}}
    (\mrm{GeLU}(x + y) - \mrm{GeLU}(x) - \mrm{GeLU}(y))
    = xy + \mcal{O}(x^3 + y^3)
\end{align*}
wherein the authors note that
\begin{align*}
    \mrm{GeLU}(x) &= \frac{x}{2} + \sqrt{\frac{2}{\pi}}x^2 + \mcal{O}(x^3) \\
    \mrm{GeLU}(x + y) &- \mrm{GeLU}(x) - \mrm{GeLU}(y) = \sqrt{\frac{2}{\pi}} xy + \mcal{O}(x^3 + y^3)
\end{align*}



