%-----------------------------------
%
\chapter{Self-Attention Layers}
%
%-----------------------------------

\section{Trivial Cases}

Identity function: zero values, use residual connection

Uniform attention: zero queries and/or keys

\section{Counting}
\label{sec:attn_counting}

\section{Table Lookup}

Given that position $i$ stores some integer $a_i$, a table lookup is a self-attention layer in which each position $i$ attends to position $a_i$.
If $f \colon [n] \to \R$ is any function and we can use the position embedding to store $f(j)$ at position $j$, then table lookup makes it possible to compute $f(a_i)$ at each position $i$.

\iffalse
\subsection{Single query}

If every position has the same query ($q^{(i)} = q$ for all $i$) and there is a minimum distance $\delta$ between all keys ($|k^{(i)}-k^{(j)}| \ge \delta$ for all $i,j$), then we can just use a FFNN to compute $\mathbb{I}[k^{(i)} = q^{(i)}] \delta$, which is piecewise linear \citep{chiang-cholak-2022-parity}:
\begin{center}
  \begin{tikzpicture}
    \begin{axis}[xtick={-1,0,1},xticklabels={$-\delta$,$0$,$\delta$,ytick={0,1},yticklabels={$0$,$\delta$}}}]
      \addplot[mark=none] coordinates { (-5,0) (-1,0) (0,1) (1,0) (5,0) };
    \end{axis}
  \end{tikzpicture}
\end{center}
(Since this doesn't use attention, it doesn't strictly belong in this chapter.)
\fi

\subsection{Layernorm hash}

\Citet{merrill-sabharwal-2024-cot} store integer $a_i$ as
\begin{align*}
  \textnormal{lh}(a_i) &= \textnormal{LayerNorm}\left(\begin{bmatrix} a_i \\ 1 \\ -a_i \\ -1 \end{bmatrix}\right) \\
  &= \frac1{\sqrt{2a_i^2+2}} \begin{bmatrix} a_i \\ 1 \\ -a_i \\ -1 \end{bmatrix}.
\end{align*}
One advantage of this representation is that if we're only able to compute $a_i/i$ and $1/i$ (as is common when counting positions using uniform attention), we can still compute $\textnormal{lh}(a_i)$. Another advantage is that $\textnormal{lh}(a)\cdot\textnormal{lh}(b)$ is uniquely maximized when $a=b$. That is, for any $i>0$, we have that $\textnormal{lh}(a_i)=\textnormal{LayerNorm}(a_i/i,1/i,-a_i/i,-1/i) = \textnormal{LayerNorm}(a_i,1,-a_i,-1)$. So if we let
\begin{align*}
  q_i &= \textnormal{lh}(a_i) \\
  k_j &= \textnormal{lh}(j)
\end{align*}
then the dot product $s_{i,j} = \textnormal{lh}(a_i) \cdot \textnormal{lh}(j)$ is uniquely maximized when $a_i = j$, so that self-attention will retrieve the value $v_j$ such that $a_i = j$. In the setting of hard-attention, this allows us to select only the position vector at position $j=a_i$, after which the the value of $f(j)$ can be retrieved by appropriately setting $W^{(V)}$.

The disadvantage of this representation is that it requires the ability to selectively apply $\textnormal{LayerNorm}$ to just four components of a vector. We can't do this with standard transformer layers (either pre-norm or post-norm), but we can if we use pre-norm \emph{and} insert a linear transformation $\mathbf{W}^{(\textnormal{N})}$ before the layer normalization:
\begin{align*}
  \mathbf{y} &= \textnormal{Sublayer}(\textnormal{LayerNorm}(\mathbf{W}^{(\textnormal{N})} \mathbf{x})) + \mathbf{x}.
\end{align*}

\subsection{Quadratic maximization}

\Citet{barcelo-etal-2024-logical} include $j$ and $j^2$ in the position embedding, allowing table lookup as follows:
\begin{align*}
  q_i &= \begin{bmatrix} a_i \\ 1 \end{bmatrix} \\
  k_j &= \begin{bmatrix} 2j \\ -j^2 \end{bmatrix}
\end{align*}
The dot product $s_{i,j}$ is $2a_ij - j^2$, which is uniquely maximized when $j=a_i$.
This is true even if either $q_i$ or $k_j$ is scaled by some factor.

\iffalse
\subsection{$-|\text{Dot-product}|$ attention}

\citep{perez-etal-2021-turing}
\fi

\section{Dot-Product and Disjunctive Normal Form}
\label{sec:att_dnf}

\section{Tie Breaking}

When using \emph{average-hard} attention, sometimes we want to be able to simulate \emph{leftmost-hard} or \emph{rightmost-hard} attention.
